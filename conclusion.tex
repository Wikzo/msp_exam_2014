\chapter{Conclusion, Discussion and Future Development}
The concept of "being in flow" is quite interesting and something designers can potentially strive for by making use of intrinsic rewards, as well as aiming to implement systems that fulfil the characteristics mentioned in Section \ref{char}. That being said, flow is a subjective feeling that can be hard to obtain.

It was expected that it would be difficult to measure any kind of flow state in the span of the short prototype tested on the eight participants. There was simply not enough time to get any indications of participants feeling flow.

That being said, it would be interesting to see how much additional time and effort would be needed for the participants to feel flow. For this to happen, it seems arguable that the prototype/game would need a lot more depth in its gameplay to obtain/maintain any kind of flow. An important element to consider is the difficulty of the gameplay: it has to scale with the players' abilities. Over time, people would become better at the game, meaning that they would gradually require greater challenges to not become bored. Since the game is in the multiplayer genre, the other players provide the difficulty. It would be interesting to see the development of this, since some players would grow faster than others. This would potentially create unbalance, where players are significantly better/worse than their opponents. One could imagine that the game would need additional systems to maintaning the balance between the players, e.g., by providing handicapping compensations/advantages, in order for all players to feel equally engaged and challenged. More research is needed to examine how this would affect players' flow states.