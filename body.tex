\chapter{Body of Argument}
For this paper, the following question has been chosen to examine:

\begin{fancyquotes}
Question 6: Flow State and Intrinsic Motivation Inventory Questionnaires
Design a set of questions based on Flow State and Intrinsic Motivation Inventory Questionnaires for your project/larger project as discussed in Lecture 8.

Discuss the potential cross-overs of the two systems with the items you have chosen, the rationales behind the chosen items and the process for customization of the items to address the identified queries for your own project. Implement and/or design an experimental future set up for your project where you would implement these questionnaires and discuss expected/actual outcomes. 

\end{fancyquotes}

For this task, a paper entitled \textit{Intrinsic Motivation Inventory (IMI)} (from now on referred to as \citep{imiOne}) has been provided. This will form the basis of the questions that will be formulated in the questionnaires.

According to \cite{imiTwo}, the intrinsic motivation inventory} has gained widespread acceptance as a way to measure intrinsic motivation in the context of sport and exercise. It determines an individual's level of intrinsic motivation as an addedive [sic] function of four underlying dimensions. These are: perceived competence, interest-enjoyment, pressure-tension, and effort-importance \citep{imiTwo}. IMI consists of a big number of questions; however, the full set is rarely used, and it seems like inclusion/exclusion of any one factor does not affect the properties of the remaining factors \citep{imiTwo}. They can easily be modified to suit a specific activity \citep{imiOne}.