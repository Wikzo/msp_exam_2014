\chapter{Analysis and Findings}
To be able to analyze and conclude statistically on any sets of data, one needs a larger sample size than what was achieved in this small test. A test is considered large when it has 30 or more samples \citep{statisticsBook}. Since the test here only consisted of eight participants, it is difficult to make any proper conclusions. However, I've chosen to look at some general tendencies, based on the data from the tests. In the following, I will only look at data related to the flow and IMI models.

When looking at the flow-related data, it seems that participants had a decent amount of focus on the game. This is the first requirement in order to to achieve the flow state, so it's important that participants felt present and focused on what they were doing. Overall, they graded themselves highly on the statements \textit{I was completely focused on the task at hand} and \textit{I did things spontaneously and automatically without having to think}. However, they graded themselves on average on the \textit{It was no effort to keep my mind on what was happening statement}. This might indicate that some things could be improved to obtain proper focus in the game.

Another important aspect is that participants lost their reflective self-consciousness, i.e., that they lost their awareness as being social actors. According to the \textit{I was not concerned with how others may have been evaluating me} statement, this was a success, since most of the participants graded themselves highly on this.

For the next statements that cover the aspect of feeling in control and feeling that one's skills match the given challenges, it seems that things could be improved upon. This is most likely due to the state of the prototype and its lack of feedback in certain areas. This was expected, since many of the elements worked using the Wizard of Oz method, where things in general feel slower and less responsive compared to a fully-fledged product. This made it particular difficult for participants to achieve any kind of flow-like state.

Due to the low amount of actual questions asked in the questionnaire about Intrinsic Motivation Inventory, it is hard to spot any significant tendencies. In general, participants responded very differently, often using the whole scale from 1 to 5, which could potentially mean that some of the questions were either confusing or didn't fit the game prototype very well. Some of the questions might have been too vague, especially when considering the early state of the prototype.

In general, it appears that participants didn't feel pressured or nervous. Also, it seems that the game didn't have an influence of how much/little participants trusted each other. This was to expect, since none of the deceptive game mechanics had yet to be put into the prototype.

Additionally, when asked to describe their experience with the game, participants responded with words such as \textit{fun}, \textit{challenging}, \textit{hectic}, \textit{immersive}, \textit{social}, \textit{active}, \textit{creative}, \textit{engaging} and \textit{exciting}.