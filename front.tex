\thispagestyle{empty} %fjerner sidetal 
\hspace{6cm} \vspace{0.1cm}
\begin{center}
\textbf{\huge {MSP Exam Mini Paper:\\Flow and Intrinsic Motivation Inventory}}\\ \vspace{1cm}

%\vspace{1cm}
\Large{\textbf{Gustav Dahl, Study nr. 20113263}}

\textbf{ABSTRACT}

Flow is a an optimal state humans can achieve. It gives a sense of ecstasy, where one has intense focus on the present moment and loses sense of time and self-consciousness. To obtain this state, an activity must fulfil a set of requirements to give the best experience possible. There should be a fine balance between the participant's skills and the difficulty of the activity. Furthermore, goals and feedback should be immediate and clear. By using a set of questions, it is possible to measure flow. For this mini paper, eight test participants played a simple platforming game. Afterwards, they were presented with a questionnaire to examine their state of flow and intrinsic motivation. Due to the limited test sample, it is difficult to make any proper conclusions. One interesting aspect for future work is to analyze how flow can be scaled in a multiplayer game context where human opponents gradually increase their skill sets. Since players don't develop at the same rate, this can introduce unbalances that needs to be accounted for in the gameplay in order to obtain the flow state.

\end{center}
\vfill
Medialogy 6th semester\\
Media Sociology and Psychology\\
Aalborg University - \today